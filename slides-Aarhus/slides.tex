\documentclass[xcolor=dvipsnames]{beamer}

\usepackage{amsthm}
\usepackage{tikz}
\usepackage{amssymb}
\usepackage{amsmath}
\usepackage{amsthm}
\usepackage{lstcoq}
\usepackage{mathpartir, pftools}
\usepackage{stmaryrd}

\usepackage{tikz-cd}

\definecolor{DarkGreen}{rgb}{0.2578,0.50,0.16406}
\definecolor{Sepia}{rgb}{0.3686,0.1490,0.0706}

\tikzstyle{every picture}+=[remember picture]

\newtheorem{conjecture}{Conjecture}

\begin{document}

\title{Cumulative Inductive Types in Coq}
\author{
\underline{Amin Timany}\inst{1}
\and
Matthieu Sozeau\inst{2}
\and
Bart Jacobs\inst{1}
}
\institute{
imec-Distrinet, KU Leuven, Belgium
\and
Inria Paris \& IRIF, France
}
\date{September 18th, 2017 \\[1em] Aarhus university, Denmark}

\setbeamertemplate{navigation symbols}{}

\begin{frame}[t]
\titlepage
\end{frame}

\setbeamertemplate{navigation symbols}{\insertframenumber}
\setcounter{framenumber}{0}

\section{Universe polymorphism and Inductive Types}

\begin{frame}[t]
\begin{itemize}
\item In higher order dependent type theories:
\begin{itemize}
\item Types are also terms and hence have a type
\item Type of all types, as it should be the type of itself, leads to paradoxes, like Girard's paradox
\item Thus, we have a countably infinite hierarchy of universes (types of types):
\[
\text{\Coqe|Type|}_0, \text{\Coqe|Type|}_1, \text{\Coqe|Type|}_2, \dots
\]
where:
\[
\text{\Coqe|Type|}_0: \text{\Coqe|Type|}_1, \text{\Coqe|Type|}_1 : \text{\Coqe|Type|}_2, \dots
\]
\end{itemize}
\pause
\item Such a system is cumulative if for any type $T$ and $i$:
\[
T : \text{\Coqe|Type|}_i \Rightarrow T : \text{\Coqe|Type|}_{i+1}
\]
\item Example: Predicative Calculus of Inductive Constructions (pCIC), the logic of the proof assistant Coq
\end{itemize}
\end{frame}

\begin{frame}[t, fragile]
\begin{itemize}
\item pCIC has recently been extended with universe polymorphism
\begin{itemize}
\item Definitions can be polymorphic in universe levels, e.g., categories:
\begin{Coq}
Record Category@{i j} : Type$$@{max(i+1, j+1)} :=
  { $\color{MidnightBlue}\mathtt{Obj}$ : Type$$@{i};
    $\color{MidnightBlue}\mathtt{Hom}$ : Obj -> Obj -> Type$$@{j}; $\dots$ }.
\end{Coq}
\pause
\item To keep consistent, universe polymorphic definitions come with constraints, e.g., category of categories:
\begin{Coq}
Definition Cat@{i j k l} :=
  {| $\color{MidnightBlue}\mathtt{Obj}$ := Category@{k l};
     $\color{MidnightBlue}\mathtt{Hom}$ := fun C D => Functor@{k l k l} C D; $\dots$ |}
  : Category@{i j}.
\end{Coq}
with constraints:
\[
k < i ~\text{ and }~ l < i
\]
\end{itemize}
\end{itemize}
\end{frame}

\begin{frame}[t, fragile]
\begin{itemize}
\item For universe polymorphic inductive types, e.g., \Coqe|Category|, copies are considered
\item With no cumulativity (subtyping), i.e.,
\\ \Coqe|Category@{i j} $\preceq$ Category@{k l}| implies $\mathtt{i = k} ~\text{ and }~\mathtt{j = l}$
\pause
\item This means \Coqe|Cat@{i j k l}| is the category of all categories at \Coqe|{k l}| and \emph{not lower}\only<2->{\;\footnote{There are however categories isomorphic to the categories in lower levels.}}
\end{itemize}
\end{frame}

\begin{frame}[t, fragile]
\begin{itemize}
\item Constraints on statements about universe polymorphic inductive definitions restrict to which copies they apply
\item For \Coqe|Cat@{i j k l}| the fact that it has exponentials has constraints $\mathtt{j = k = l}$
\pause
\item In particular:
\begin{Coq}
Definition Type_Cat@{i j} :=
  {| $\color{MidnightBlue}\mathtt{Obj}$ := Type$$@{j};
     $\color{MidnightBlue}\mathtt{Hom}$ := fun A B => A -> B; $\dots$ |} : Category@{i j}.
\end{Coq}
with constraints: $j < i$
\item It is \emph{not} an object of any copy of \Coqe|Cat| with exponentials!
\pause
\item Yoneda embedding can't be simply defined as the exponential transpose of the $hom$ functor
\end{itemize}
\end{frame}

\begin{frame}[t, fragile]
\begin{itemize}
\item Inductive typs in pCIC:
\begin{mathpar}
\inferH{Ind}{
A \in Ar(s) \and \Gamma \vdash A: s' \and
\Gamma, X: A \vdash C_i: s \and C_i \in Co(X)}
{\Gamma \vdash \mathsf{Ind}(X: A)\{C_1,\dots,C_n\}: A}
\end{mathpar}
\[
Ar(s) ~\text{ is the set of types of the form: }~ \Pi \overset{\rightarrow}{x}: \overset{\rightarrow}{M}.~s
\]
\[
Co(X) ~\text{ is the set of types of the form: }~ \Pi\overset{\rightarrow}{x} : \overset{\rightarrow}{M}.~X~\overset{\rightarrow}{m}
\]
\pause
No Parameters (\Coqe|T| in \Coqe|vec T n|) are considered in this rule.
\begin{Coq}
Inductive vec (T : Type) : nat -> Type := nil : vec T 0 
  | cons : forall n, T -> vec T n -> vec T (S n).
\end{Coq}
\pause
\item Some cumulativity rules in pCIC:
\begin{mathpar}
\inferH{Conv}{
\Gamma \vdash t: A \and \Gamma \vdash B: s \and A \preceq B}
{\Gamma \vdash t: B}
\end{mathpar}
\begin{mathpar}
\inferH{C-Type}{i \le j}
{{\color{red}\mathtt{Type}}_i \preceq {\color{red}\mathtt{Type}}_j}
\and
\inferH{C-Prod}{A \simeq A' \and B \preceq B'}{\Pi x : A.~B \preceq \Pi x: A'.~B'}
\end{mathpar}
\end{itemize}
\end{frame}

\begin{frame}[t, fragile]
\begin{itemize}
\item Predicative Calculus of Cumulative Inductive Types (pCuIC):
{\small
\only<1, 6->{
\begin{mathpar}
\inferH{C-Ind}{
I \equiv (\mathsf{Ind}
(X : \Pi \vec{x} : \vec{N}.~s)\{
\Pi\vec{x_1} : \vec{M_1}.
~X~\vec{m_1},\dots,\Pi \vec{x_n} :
\vec{M_n}.~X~\vec{m_n}\})\\\
\and
I' \equiv (\mathsf{Ind}
(X : \Pi \vec{x} : \vec{N'}.~s')\{
\Pi \vec{x_1} : \vec{M_1'}.~
X~\vec{m_1'},\dots,\Pi \vec{x_n} :
\vec{M_n'}.~X~\vec{m_n'}\})\\\
\and
\forall i.~ N_i \preceq {N'}_i
\and
\forall i,j.~ (M_i)_j \preceq (M_i')_j \\\
\and
length(\vec{m}) = length(\vec{x})
\and
\forall i.~X~\vec{m_i} \simeq X~ \vec{m_i'}
}{
I~\vec{m}
\preceq
I'~\vec{m}
}
\end{mathpar}
}
\only<2>{
\begin{mathpar}
\inferH{C-Ind}{
I \equiv (\mathsf{Ind}
(X : \Pi \vec{x} : \vec{N}.~s)\{
\Pi\vec{x_1} : \vec{M_1}.
~X~\vec{m_1},\dots,\Pi \vec{x_n} :
\vec{M_n}.~X~\vec{m_n}\})\\\
\and
I' \equiv (\mathsf{Ind}
(X : \Pi \vec{x} : \vec{N'}.~s')\{
\Pi \vec{x_1} : \vec{M_1'}.~
X~\vec{m_1'},\dots,\Pi \vec{x_n} :
\vec{M_n'}.~X~\vec{m_n'}\})\\\
\and
\colorbox{SeaGreen}{$\forall i.~ N_i \preceq {N'}_i$}
\and
\forall i,j.~ (M_i)_j \preceq (M_i')_j \\\
\and
length(\vec{m}) = length(\vec{x})
\and
\forall i.~X~\vec{m_i} \simeq X~ \vec{m_i'}
}{
I~\vec{m}
\preceq
I'~\vec{m}
}
\end{mathpar}
}
\only<3>{
\begin{mathpar}
\inferH{C-Ind}{
I \equiv (\mathsf{Ind}
(X : \Pi \vec{x} : \vec{N}.~s)\{
\Pi\vec{x_1} : \vec{M_1}.
~X~\vec{m_1},\dots,\Pi \vec{x_n} :
\vec{M_n}.~X~\vec{m_n}\})\\\
\and
I' \equiv (\mathsf{Ind}
(X : \Pi \vec{x} : \vec{N'}.~s')\{
\Pi \vec{x_1} : \vec{M_1'}.~
X~\vec{m_1'},\dots,\Pi \vec{x_n} :
\vec{M_n'}.~X~\vec{m_n'}\})\\\
\and
\forall i.~ N_i \preceq {N'}_i
\and
\colorbox{SeaGreen}{$\forall i,j.~ (M_i)_j \preceq (M_i')_j$} \\\
\and
length(\vec{m}) = length(\vec{x})
\and
\forall i.~X~\vec{m_i} \simeq X~ \vec{m_i'}
}{
I~\vec{m}
\preceq
I'~\vec{m}
}
\end{mathpar}
}
\only<4>{
\begin{mathpar}
\inferH{C-Ind}{
I \equiv (\mathsf{Ind}
(X : \Pi \vec{x} : \vec{N}.~s)\{
\Pi\vec{x_1} : \vec{M_1}.
~X~\vec{m_1},\dots,\Pi \vec{x_n} :
\vec{M_n}.~X~\vec{m_n}\})\\\
\and
I' \equiv (\mathsf{Ind}
(X : \Pi \vec{x} : \vec{N'}.~s')\{
\Pi \vec{x_1} : \vec{M_1'}.~
X~\vec{m_1'},\dots,\Pi \vec{x_n} :
\vec{M_n'}.~X~\vec{m_n'}\})\\\
\and
\forall i.~ N_i \preceq {N'}_i
\and
\forall i,j.~ (M_i)_j \preceq (M_i')_j \\\
\and
length(\vec{m}) = length(\vec{x})
\and
\colorbox{SeaGreen}{$\forall i.~X~\vec{m_i} \simeq X~ \vec{m_i'}$}
}{
I~\vec{m}
\preceq
I'~\vec{m}
}
\end{mathpar}
}
\only<5>{
\begin{mathpar}
\inferH{C-Ind}{
I \equiv (\mathsf{Ind}
(X : \Pi \vec{x} : \vec{N}.~s)\{
\Pi\vec{x_1} : \vec{M_1}.
~X~\vec{m_1},\dots,\Pi \vec{x_n} :
\vec{M_n}.~X~\vec{m_n}\})\\\
\and
I' \equiv (\mathsf{Ind}
(X : \Pi \vec{x} : \vec{N'}.~s')\{
\Pi \vec{x_1} : \vec{M_1'}.~
X~\vec{m_1'},\dots,\Pi \vec{x_n} :
\vec{M_n'}.~X~\vec{m_n'}\})\\\
\and
\forall i.~ N_i \preceq {N'}_i
\and
\forall i,j.~ (M_i)_j \preceq (M_i')_j \\\
\and
\colorbox{SeaGreen}{$length(\vec{m}) = length(\vec{x})$}
\and
\forall i.~X~\vec{m_i} \simeq X~ \vec{m_i'}
}{
I~\vec{m}
\preceq
I'~\vec{m}
}
\end{mathpar}
}
}
\pause
\pause
\pause
\pause
\pause
\pause
\item Example:
\begin{center}
\Coqe|Category@{i j}| $\equiv \mathsf{Ind}(X: \text{\Coqe|Type|}_{max(i+1,j+1)})\{\Pi o: \text{\Coqe|Type|}_i.\Pi h: o \to o \to \text{\Coqe|Type|}_j.\cdots\}$
\end{center}
\item By \textsc{C-Ind}:
\begin{center}
$\mathtt{i \le k}$ and $\mathtt{j \le l}$ $\Rightarrow$ \Coqe|Category@{i j}| $\preceq$ \Coqe|Category@{k l}|
\end{center}
\pause
\item Notice \textsc{C-Ind} does not consider parameters or sort of the inductive type
\end{itemize}
\end{frame}

\begin{frame}[fragile]
\begin{itemize}
\item For \Coqe|Cat@{i j k l}| with exponentials we had the constraints: $\mathtt{j = k = l}$

\item \Coqe|Type_Cat@{i' j'}| we had the constraint: $j' < i'$

\item Now \Coqe|Type_Cat@{i' j'} : Obj Cat@{i j k l}| just imposes the constraint:
$\mathtt{i' \le k \land j' \le l}$

\end{itemize}
\end{frame}

\begin{frame}[t, fragile]
\begin{itemize}
\item Example:
\begin{center}
\Coqe|list@{i} (A : Type$_i$)| $\equiv \mathsf{Ind}(X: \text{\Coqe|Type|$_{i}$})\{X, A \to X \to X\}$
\end{center}
\item By \textsc{C-Ind}:
\begin{center}
\Coqe|list@{i} A| $\preceq$ \Coqe|list@{j} A| \pause \hspace{1em} (\textbf{regardless of $\mathtt{i}$ and $\mathtt{j}$})
\end{center}
\pause
\item In pCuIC we consider \emph{fully applied} inductive types $I~\vec{m}$ and $I'~\vec{m}$ convertible if they are mutually subtypes
\begin{mathpar}
\inferH{Conv-Ind}{I~\vec{m} \preceq I'~\vec{m} \and I'~\vec{m} \preceq I~\vec{m}}{I~\vec{m} \simeq I'~\vec{m}}
\end{mathpar}
\pause
\item Examples:
\begin{center}
$\mathtt{i = k}$ and $\mathtt{j = l}$ $\Rightarrow$ \Coqe|Category@{i j}| $\simeq$ \Coqe|Category@{k l}|
\end{center}
\begin{center}
\Coqe|list@{i} A| $\simeq$ \Coqe|list@{j} A| \hspace{1em} (\textbf{regardless of $\mathtt{i}$ and $\mathtt{j}$})
\end{center}
\end{itemize}
\end{frame}

\begin{frame}{Demo}
\begin{center}
This is implemented in Coq!
\end{center}
\end{frame}

\newcommand{\sem}[1]{\llbracket #1 \rrbracket}

\begin{frame}[t]{Theoretical justification}
\begin{itemize}
\item We construct a set theoretic model for pCuIC $\sem{\cdot}: \mathit{Terms}_\mathit{pCuIC} \to ZFC$\footnote{ZFC with suitable axioms, e.g., inaccessible cardinals, to model pCuIC universes}
\item For subtyping $A \preceq B$ we have $\sem{A} \subseteq \sem{B}$
\item Inductive types interpreted using least fixpoints of \textbf{monotone}\footnote{Due to strict positivity condition} functions
\item This justifies both \textsc{C-Ind} and \textsc{Conv-Ind}
\end{itemize}
\end{frame}

\begin{frame}
\centering
\Huge Thanks
\end{frame}

\end{document}

