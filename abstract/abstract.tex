\documentclass{easychair}
\usepackage{color}
\usepackage{lstcoq}
\usepackage{todonotes}
\usepackage{mathpartir, pftools}
\usepackage{qsymbols}

\title{Cumulative inductive types in Coq}

% Authors are joined by \and. Their affiliations are given by \inst, which indexes
% into the list defined using \institute
%
\author{
% Serguei A. Mokhov\inst{1}\thanks{Designed and implemented the class style}
% \and
%     Geoff Sutcliffe\inst{2}\thanks{Did numerous tests and provided a lot of suggestions}
% \and
%    Andrei Voronkov\inst{3}\inst{4}\inst{5}\thanks{Masterminded EasyChair and created versions
%      3.0--3.4 of the class style}
}

% Institutes for affiliations are also joined by \and,
\institute{
%   Concordia University,
%   Montreal, Quebec, Canada\\
%   \email{mokhov@cse.concordia.ca}
% \and
%    University of Miami,
%    Miami, Florida, U.S.A.\\
%    \email{geoff@cs.miami.edu}\\
% \and
%    University of Manchester,
%    Manchester, U.K.\\
%    \email{andrei@voronkov.com}\\
% \and
%    Chalmers University of Technology,
%    Gothenburg, Sweden
% \and
%    EasyChair
 }

%  \authorrunning{} has to be set for the shorter version of the authors' names;
% otherwise a warning will be rendered in the running heads. When processed by
% EasyChair, this command is mandatory: a document without \authorrunning
% will be rejected by EasyChair

\authorrunning{% Mokhov, Sutcliffe and Voronkov
}

% \titlerunning{} has to be set to either the main title or its shorter
% version for the running heads. When processed by
% EasyChair, this command is mandatory: a document without \titlerunning
% will be rejected by EasyChair
\titlerunning{Cumulative inductive types in Coq}

\newcommand{\Type}[1]{{\color{red} \mathtt{Type}}_{#1}}

\begin{document}

\maketitle

In order to avoid well-know paradoxes associated with self-referential
definitions higher-order dependent type theories usually use a
countably infinite hierarchy of universes (also known as sorts),
$\Type{0} : \Type{1} : \cdots$. Such type systems are called
cumulative if for any type $T$ we have that $T : \Type{i}$ implies
$T : \Type{i+1}$. Predicative calculus of inductive constructions
(pCIC), the underlying logic of the Coq proof assistant is one such
system.\todo{cite pCIC paper!?}

Earlier work \cite{DBLP:conf/itp/SozeauT14} on universe polymorphism
in Coq allows constructions to be polymorphic in universe levels.  The
quintessential universe polymorphic construction is the polymorphic definition of
categories:
\Coqe|Record Category$_{\mathtt{i, j}}$ := {Obj : $\Type{\mathtt{i}}$; Hom : Obj -> Obj -> $\Type{\mathtt{j}}$; $\cdots$}|.\footnotemark{}
\footnotetext{Records in Coq are syntactic sugar for an inductive type
with a single constructor.}

However, pCIC does not extend the subtyping relation (induced by
cumulativity) to inductive types. As a result there is no subtyping
relation between instances of a universe polymorphic inductive type.
That is for a category \Coqe|C|, having both
\Coqe|C : Category$_{\mathtt{i, j}}$| and \Coqe|C : Category$_{\mathtt{i', j'}}$|
is only possible if $\mathtt{i = i'}$ and $\mathtt{j = j'}$.
\todo[inline]{should we elaborate a bit that we want subtyping for categories because smallness/largeness!?}

In previous work Timany et al. \cite{DBLP:conf/ictac/Timany015} extend
pCIC to pCuIC (predicative Calculus of Cumulative Inductive
Constructions). This is essentially the system pCIC with a single
subtyping added to it:
\begin{mathpar}
\inferH{C-Ind}{
I \equiv (\mathsf{Ind}
(X : \Pi \vec{x} : \vec{N}.~s)\{
\Pi\vec{x_1} : \vec{M_1}.
~X~\vec{m_1},\dots,\Pi \vec{x_n} :
\vec{M_n}.~X~\vec{m_n}\})\\\
\and
I' \equiv (\mathsf{Ind}
(X : \Pi \vec{x} : \vec{N'}.~s')\{
\Pi \vec{x_1} : \vec{M_1'}.~
X~\vec{m_1'},\dots,\Pi \vec{x_n} :
\vec{M_n'}.~X~\vec{m_n'}\})\\\
\and
s \preceq s'
\and
\forall i.~ N_i \preceq {N'}_i
\and
\forall i,j.~ (M_i)_j \preceq (M_i')_j\\\
\and
length(\vec{m}) = length(\vec{x})
\and
\forall i.~X~\vec{m_i} \simeq X~ \vec{m_i'}
}{
I~\vec{m}
\preceq
I'~\vec{m}
}
\end{mathpar}
The two terms $I$ and $I'$ are two inductive definitions (type
constructors\footnote{Not to be confused with constructors of
  inductive types}) with indexes with types $\vec{N}$ and $\vec{N'}$
respectively. They are respectively in sorts (universes) $s$ and $s'$.
They each have $n$ constructor with $i$\textsuperscript{th}
constructors are of the type $\Pi\vec{x_i} : \vec{M_i}.~X~\vec{m_i}$
and $\Pi\vec{x_i} : \vec{M_i'}.~X~\vec{m_i'}$ for $I$ and $I'$
respectively. With this out of the way, the reading of the rule
\ref{C-Ind} is now straightforward. The type $I~\vec{m}$ is a subtype
of the type $I'~\vec{m}$ if the corresponding arguments of
corresponding constructors in $I$ are sub types of those of $I'$.  In
other words, if the terms $\vec{v}$ can be applied to the
$i$\textsuperscript{th} of $I$ to construct a term of type $I~\vec{m}$
then the same terms $\vec{v}$ can be applied to the corresponding
constructor of $I'$ to construct a term of type $I'~\vec{m}$.  Notice
in \ref{C-Ind} the indexes of the compared inductive types $\vec{m}$
do not appear anywhere above the line. Indeed the particular instance
of the inductive type is not important when comparing two (dependent)
inductive types. The terms $\vec{m}$ are added because subtyping of
inductive types allows only the comparison of two fully applied type
constructors, i.e., two types.

Using the rule \ref{C-Ind} above (in presence of universe polymorphism)
we can derive
\Coqe|Category$_{\mathtt{i, j}}$| $\preceq$ \Coqe|Category$_{\mathtt{i', j'}}$|
whenever $\mathtt{i \leq i'}$ and $\mathtt{j \leq j'}$.


\def\vec#1{\overset{\rightarrow}{#1}}



- describe the system (just one additional inference rule)

\subsection*{Template Polymorphism}

Before the work of \cite{SozeauT14} on adding full universe polymorphism
to Coq, the system enjoyed a restricted form of polymorphism for
inductive types, which we coined template polymorphism. The idea is to
give more precise types to applications of inductive types to their
parameters, so that e.g. the inferred type of \texttt{list nat} is
$\Type{0}$ instead of $\Type{i}$ for a global type level $i$.

Technically, consider an inductive type $I$ of arity
$\forall \vec{P}, \vec{A} "->" s$ where $\vec{P}$ are parameters and
$\vec{A}$ the indices.  When the type of the $n$-th parameter is
$\Type{l}$ for some level $l$ and $l$ occurs in the sort $s$, the
inductive is made parametric on $l$. When we infer the type of an
application of $I$ to parameters $\vec{p}$, we compute its type as
$\forall \vec{A} "->" s[l'/l]$ where $p_n : \Type{l'}$, i.e using the
actual inferred types of the parameters.


\subsection*{Motivating examples}

- example of complete + small => preorder, show how lifting + rule for
cumulativity explain the impossibility of applying this to Types@{i}.

\subsection*{Consistency and Strong Normalization}
- idea of the consistency proof using Werner \& Lee's model

- for SN, ???

\bibliographystyle{plain}
\bibliography{bib}
\end{document}


%  LocalWords:  cumulativity polymorphism Coq pCIC subtyping Timany
%  LocalWords:  pCuIC subtype
