\documentclass{easychair}
\usepackage{color}
\usepackage{lstcoq}
\usepackage{todonotes}
\usepackage{mathpartir, pftools}
\usepackage{qsymbols}
\usepackage{natbib}

\title{Cumulative inductive types in Coq}

% Authors are joined by \and. Their affiliations are given by \inst, which indexes
% into the list defined using \institute
%
\author{
% Serguei A. Mokhov\inst{1}\thanks{Designed and implemented the class style}
% \and
%     Geoff Sutcliffe\inst{2}\thanks{Did numerous tests and provided a lot of suggestions}
% \and
%    Andrei Voronkov\inst{3}\inst{4}\inst{5}\thanks{Masterminded EasyChair and created versions
%      3.0--3.4 of the class style}
}

% Institutes for affiliations are also joined by \and,
\institute{
%   Concordia University,
%   Montreal, Quebec, Canada\\
%   \email{mokhov@cse.concordia.ca}
% \and
%    University of Miami,
%    Miami, Florida, U.S.A.\\
%    \email{geoff@cs.miami.edu}\\
% \and
%    University of Manchester,
%    Manchester, U.K.\\
%    \email{andrei@voronkov.com}\\
% \and
%    Chalmers University of Technology,
%    Gothenburg, Sweden
% \and
%    EasyChair
 }

%  \authorrunning{} has to be set for the shorter version of the authors' names;
% otherwise a warning will be rendered in the running heads. When processed by
% EasyChair, this command is mandatory: a document without \authorrunning
% will be rejected by EasyChair

\authorrunning{% Mokhov, Sutcliffe and Voronkov
}

% \titlerunning{} has to be set to either the main title or its shorter
% version for the running heads. When processed by
% EasyChair, this command is mandatory: a document without \titlerunning
% will be rejected by EasyChair
\titlerunning{Cumulative inductive types in Coq}

\newcommand{\Type}[1]{{\color{red} \mathtt{Type}}_{#1}}
\newcommand{\Set}{{\color{red} \mathtt{Set}}}
\begin{document}

\maketitle

In order to avoid well-know paradoxes associated with self-referential
definitions higher-order dependent type theories usually use a
countably infinite hierarchy of universes (also known as sorts),
$\Set = \Type{0} : \Type{1} : \cdots$. Such type systems are called
cumulative if for any type $T$ we have that $T : \Type{i}$ implies
$T : \Type{i+1}$. Predicative calculus of inductive constructions
(pCIC) \cite{coq, DBLP:journals/corr/abs-1111-0123}, the underlying logic of the Coq proof assistant is one such
system.

Earlier work \cite{DBLP:conf/itp/SozeauT14} on universe polymorphism
in Coq allows constructions to be polymorphic in universe levels.  The
quintessential universe polymorphic construction is the polymorphic definition of
categories:
\Coqe|Record Category$_{\mathtt{i, j}}$ := {Obj : $\Type{\mathtt{i}}$; Hom : Obj -> Obj -> $\Type{\mathtt{j}}$; $\cdots$}|.\footnotemark{}
\footnotetext{Records in Coq are syntactic sugar for an inductive type
with a single constructor.}

However, pCIC does not extend the subtyping relation (induced by
cumulativity) to inductive types. As a result there is no subtyping
relation between instances of a universe polymorphic inductive type.
That is for a category \Coqe|C|, having both
\Coqe|C : Category$_{\mathtt{i, j}}$| and \Coqe|C : Category$_{\mathtt{i', j'}}$|
is only possible if $\mathtt{i = i'}$ and $\mathtt{j = j'}$.
In previous work Timany et al. \cite{DBLP:conf/ictac/Timany015} extend
pCIC to pCuIC (predicative Calculus of Cumulative Inductive
Constructions). This is essentially the system pCIC with a single
subtyping added to it:\footnotemark{}
\begin{mathpar}
\inferH{C-Ind}{
I \equiv (\mathsf{Ind}
(X : \Pi \vec{x} : \vec{N}.~s)\{
\Pi\vec{x_1} : \vec{M_1}.
~X~\vec{m_1},\dots,\Pi \vec{x_n} :
\vec{M_n}.~X~\vec{m_n}\})\\\
\and
I' \equiv (\mathsf{Ind}
(X : \Pi \vec{x} : \vec{N'}.~s')\{
\Pi \vec{x_1} : \vec{M_1'}.~
X~\vec{m_1'},\dots,\Pi \vec{x_n} :
\vec{M_n'}.~X~\vec{m_n'}\})\\\
\and
\forall i.~ N_i \preceq {N'}_i
\and
\forall i,j.~ (M_i)_j \preceq (M_i')_j
\and
length(\vec{m}) = length(\vec{x})
\and
\forall i.~X~\vec{m_i} \simeq X~ \vec{m_i'}
}{
I~\vec{m}
\preceq
I'~\vec{m}
}
\end{mathpar}
\footnotetext{The rule \ref{C-Ind} is slightly changed here so that it applies to
 template polymorphism explained below.}%
The two terms $I$ and $I'$ are two inductive definitions (type
constructors\footnote{Not to be confused with constructors of
  inductive types}) with indexes with types $\vec{N}$ and $\vec{N'}$
respectively. They are respectively in sorts (universes) $s$ and $s'$.
They each have $n$ constructor with $i$\textsuperscript{th}
constructors are of the type $\Pi\vec{x_i} : \vec{M_i}.~X~\vec{m_i}$
and $\Pi\vec{x_i} : \vec{M_i'}.~X~\vec{m_i'}$ for $I$ and $I'$
respectively. With this out of the way, the reading of the rule
\ref{C-Ind} is now straightforward. The type $I~\vec{m}$ is a subtype
of the type $I'~\vec{m}$ if the corresponding arguments of
corresponding constructors in $I$ are sub types of those of $I'$.  In
other words, if the terms $\vec{v}$ can be applied to the
$i$\textsuperscript{th} of $I$ to construct a term of type $I~\vec{m}$
then the same terms $\vec{v}$ can be applied to the corresponding
constructor of $I'$ to construct a term of type $I'~\vec{m}$.  % Notice
% in \ref{C-Ind} the indexes of the compared inductive types $\vec{m}$
% do not appear anywhere above the line. Indeed the particular instance
% of the inductive type is not important when comparing two (dependent)
% inductive types. The terms $\vec{m}$ are added because subtyping of
% inductive types allows only the comparison of two fully applied type
% constructors, i.e., two types.
Using the rule \ref{C-Ind} above (in presence of universe polymorphism)
we can derive
\Coqe|Category$_{\mathtt{i, j}}$| $\preceq$ \Coqe|Category$_{\mathtt{i', j'}}$|
whenever $\mathtt{i \le i'}$ and $\mathtt{j \le j'}$.

\def\vec#1{\overset{\rightarrow}{#1}}
\def\clist{\texttt{list}}
\def\nat{\texttt{nat}}
\def\ulist#1{\texttt{list@\{}#1\texttt{\}}}
\paragraph{Template Polymorphism}

Before the addition of full universe polymorphism to Coq, the system
enjoyed a restricted form of polymorphism for inductive types, which was
since coined template polymorphism. The idea is to give more precise
types to applications of inductive types to their parameters, so that
e.g. the inferred type of \texttt{list nat} is $\Type{0}$ instead of
$\Type{i}$ for a global type level $i$.

Technically, consider an inductive type $I$ of arity
$\forall \vec{P}, \vec{A} "->" s$ where $\vec{P}$ are parameters and
$\vec{A}$ the indices.  When the type of the $n$-th parameter is
$\Type{l}$ for some level $l$ and $l$ occurs in the sort $s$, the
inductive is made parametric on $l$. When we infer the type of an
application of $I$ to parameters $\vec{p}$, we compute its type as
$\forall \vec{A} "->" s[l'/l]$ where $p_n : \Type{l'}$, i.e using the
actual inferred types of the parameters.

This extension allows to naturally identify $\clist (\nat : \Set)$ and
$\clist (\nat : \Type{i})$ by convertibility, whereas with full universe
polymorphism when comparing to $\ulist{\Set}~(\nat : \Set)$ and
$\ulist{i}~ (\nat : \Type{i})$ with $\Set < i$ we would fail as equating
$i$ and $\Set$ is forbidden. With our new rule, this conversion will be
validated as these two $\texttt{list}$ instances become convertible.
Indeed, convertibility on inductive applications will now be defined as
cumulativity in both directions and in this case $\ulist{i}$
cumulativity imposes no constraint on its universe variable. This change
will allow a complete compatibility when moving inductive definitions
from the template polymorphic treatment to the new one.

\paragraph{Motivating examples}
The category theory library by Timany et
al. \cite{DBLP:conf/rta/Timany016} represents (relative) smallness and
largeness of categories through universe levels. Smallness and
largeness side-conditions for constructions is inferred by the kernel
of Coq. In loc. cit. the authors prove a well-known theorem stating
that any small and complete category is a preorder category. Coq
infers that this theorem would only apply to a category
\Coqe|C : Category$_{\mathtt{i, j}}$| only if $\mathtt{j \le i}$ and thus not to
the category \Coqe|Types@{i} : Category$_{\mathtt{i, i + 1}}$| of
types at level $\mathtt{i}$ (and functions between them) which is
complete but not small.  In a system with the rule \ref{C-Ind} we have
\Coqe|Types@{i} : Category$_{\mathtt{k, l}}$| for $\mathtt{i < k}$,
$\mathtt{i + 1 < l}$ and $\mathtt{l \le k}$. However, subtyping of the
notion completeness would not allow for the proof of completeness of
\Coqe|Types@{i}| to be lifted as required. Intuitively, that would
require the category to have limits of all functors from possibly larger
categories.

\paragraph{Consistency and Strong Normalization}
The model constructed for pCIC by Lee et
al. \cite{DBLP:journals/corr/abs-1111-0123} is a set theoretic model
that for inductive types considers the (fixpoints of the function
generated by) constructors applied to all applicable terms. Therefore,
the model readily includes all elements of the inductive types
including those added by the rule \ref{C-Ind}. Hence it is only
natural to expect (and it is our conjecture that) the same model
proves strong normalization for and thus the consistency of Coq when
extended with the rule \ref{C-Ind} replacing template polymorphism is
described above.
% - idea of the consistency proof using Werner \& Lee's model

% - for SN, ???

\paragraph{Implementation}
The rule \ref{C-Ind} above can be implemented in Coq very efficiently.
The idea is that as soon as we define an inductive type, we compare
two fresh instances of it (with two different sets of universe
variables) to compute the set of constraints necessary for subtyping
relation for different instances of that inductive type. Subsequent
comparisons during type checking/inference will use these constraints.
It is our plan to implement the rule \ref{C-Ind} in Coq and
accordingly eliminate the template polymorphism mechanism.
These new features are planed as part of the next major release of
the Coq proof assistant (Coq 8.7).

\setlength{\bibsep}{0pt} % or use whatever dimension you want
\renewcommand{\bibfont}{\small} % or any other  appropriate font command
\bibliographystyle{plain}
\bibliography{bib}
\end{document}


%  LocalWords:  cumulativity polymorphism Coq pCIC subtyping Timany
%  LocalWords:  pCuIC subtype preorder fixpoints
