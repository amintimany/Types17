\documentclass{easychair}
\usepackage{color}
\title{Cumulative inductive types in Coq}

% Authors are joined by \and. Their affiliations are given by \inst, which indexes
% into the list defined using \institute
%
\author{
% Serguei A. Mokhov\inst{1}\thanks{Designed and implemented the class style}
% \and
%     Geoff Sutcliffe\inst{2}\thanks{Did numerous tests and provided a lot of suggestions}
% \and
%    Andrei Voronkov\inst{3}\inst{4}\inst{5}\thanks{Masterminded EasyChair and created versions
%      3.0--3.4 of the class style}
}

% Institutes for affiliations are also joined by \and,
\institute{
%   Concordia University,
%   Montreal, Quebec, Canada\\
%   \email{mokhov@cse.concordia.ca}
% \and
%    University of Miami,
%    Miami, Florida, U.S.A.\\
%    \email{geoff@cs.miami.edu}\\
% \and
%    University of Manchester,
%    Manchester, U.K.\\
%    \email{andrei@voronkov.com}\\
% \and
%    Chalmers University of Technology,
%    Gothenburg, Sweden
% \and
%    EasyChair
 }

%  \authorrunning{} has to be set for the shorter version of the authors' names;
% otherwise a warning will be rendered in the running heads. When processed by
% EasyChair, this command is mandatory: a document without \authorrunning
% will be rejected by EasyChair

\authorrunning{% Mokhov, Sutcliffe and Voronkov
}

% \titlerunning{} has to be set to either the main title or its shorter
% version for the running heads. When processed by
% EasyChair, this command is mandatory: a document without \titlerunning
% will be rejected by EasyChair
\titlerunning{Cumulative inductive types in Coq}

\newcommand{\Type}[1]{{\color{red} \mathtt{Type}}_{#1}}

\begin{document}

\maketitle

In order to avoid well-know paradoxes associated with self-referential
definitions higher-order dependent type theories usually use a
countably infinite hierarchy of universes (also known as sorts),
$\Type{0} : \Type{1} : \cdots$. Such type systems are called
cumulative if for any type $T$ we have that $T : \Type{i}$ implies
$T : \Type{i+1}$. Calculus of inductive constructions (CIC), the
underlying logic of the Coq proof assistant is one such system.

Earlier work \cite{DBLP:conf/itp/SozeauT14} on universe polymorphism
in Coq allows definitions to by polymorphic in universe levels.


- describe the system (just one additional inference rule)

- explain template polymorphism ``semantics'' (MS)

- example of complete + small => preorder, show how lifting + rule for
cumulativity explain the impossibility of applying this to Types@{i}.

- idea of the consistency proof using Werner \& Lee's model

- for SN, ???

The text

\bibliographystyle{plain}
\bibliography{bib}
\end{document}


%  LocalWords:  cumulativity polymorphism Coq CIC
